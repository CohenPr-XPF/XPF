\documentclass[12pt,]{article}
\usepackage{lmodern}
\usepackage{amssymb,amsmath}
\usepackage{ifxetex,ifluatex}
\usepackage{fixltx2e} % provides \textsubscript
\ifnum 0\ifxetex 1\fi\ifluatex 1\fi=0 % if pdftex
  \usepackage[T1]{fontenc}
  \usepackage[utf8]{inputenc}
\else % if luatex or xelatex
  \ifxetex
    \usepackage{mathspec}
  \else
    \usepackage{fontspec}
  \fi
  \defaultfontfeatures{Ligatures=TeX,Scale=MatchLowercase}
    \setmainfont[]{DejaVu Serif}
\fi
% use upquote if available, for straight quotes in verbatim environments
\IfFileExists{upquote.sty}{\usepackage{upquote}}{}
% use microtype if available
\IfFileExists{microtype.sty}{%
\usepackage{microtype}
\UseMicrotypeSet[protrusion]{basicmath} % disable protrusion for tt fonts
}{}
\usepackage[margin=1.2in]{geometry}
\usepackage{hyperref}
\hypersetup{unicode=true,
            pdftitle={Hungarian},
            pdfauthor={Shiying Yang},
            pdfborder={0 0 0},
            breaklinks=true}
\urlstyle{same}  % don't use monospace font for urls
\usepackage{longtable,booktabs}
\usepackage{graphicx,grffile}
\makeatletter
\def\maxwidth{\ifdim\Gin@nat@width>\linewidth\linewidth\else\Gin@nat@width\fi}
\def\maxheight{\ifdim\Gin@nat@height>\textheight\textheight\else\Gin@nat@height\fi}
\makeatother
% Scale images if necessary, so that they will not overflow the page
% margins by default, and it is still possible to overwrite the defaults
% using explicit options in \includegraphics[width, height, ...]{}
\setkeys{Gin}{width=\maxwidth,height=\maxheight,keepaspectratio}
\IfFileExists{parskip.sty}{%
\usepackage{parskip}
}{% else
\setlength{\parindent}{0pt}
\setlength{\parskip}{6pt plus 2pt minus 1pt}
}
\setlength{\emergencystretch}{3em}  % prevent overfull lines
\providecommand{\tightlist}{%
  \setlength{\itemsep}{0pt}\setlength{\parskip}{0pt}}
\setcounter{secnumdepth}{5}
% Redefines (sub)paragraphs to behave more like sections
\ifx\paragraph\undefined\else
\let\oldparagraph\paragraph
\renewcommand{\paragraph}[1]{\oldparagraph{#1}\mbox{}}
\fi
\ifx\subparagraph\undefined\else
\let\oldsubparagraph\subparagraph
\renewcommand{\subparagraph}[1]{\oldsubparagraph{#1}\mbox{}}
\fi

%%% Use protect on footnotes to avoid problems with footnotes in titles
\let\rmarkdownfootnote\footnote%
\def\footnote{\protect\rmarkdownfootnote}

%%% Change title format to be more compact
\usepackage{titling}

% Create subtitle command for use in maketitle
\newcommand{\subtitle}[1]{
  \posttitle{
    \begin{center}\large#1\end{center}
    }
}

\setlength{\droptitle}{-2em}

  \title{Hungarian}
    \pretitle{\vspace{\droptitle}\centering\huge}
  \posttitle{\par}
    \author{Shiying Yang}
    \preauthor{\centering\large\emph}
  \postauthor{\par}
      \predate{\centering\large\emph}
  \postdate{\par}
    \date{4/1/2019}

\usepackage{booktabs}
\usepackage{longtable}
\usepackage{array}
\usepackage{multirow}
\usepackage{wrapfig}
\usepackage{float}
\usepackage{colortbl}
\usepackage{pdflscape}
\usepackage{tabu}
\usepackage{threeparttable}
\usepackage{threeparttablex}
\usepackage[normalem]{ulem}
\usepackage{makecell}
\usepackage{xcolor}

\usepackage{setspace}\doublespacing
\usepackage{tipa}

\begin{document}
\maketitle

\hypertarget{background}{%
\section{Background}\label{background}}

\textbf{Language family:}

\begin{itemize}
\tightlist
\item
  Uralic

  \begin{itemize}
  \tightlist
  \item
    Finno-Ugric
  \end{itemize}
\end{itemize}

\textbf{Adopted variety:} ECH (Educated Colloquial Hungarian) (Siptár
and Törkenczy 2000)

\hypertarget{phonology-and-orthography}{%
\section{Phonology and Orthography}\label{phonology-and-orthography}}

\hypertarget{con}{%
\subsection{\texorpdfstring{Consonants (Siptár and Törkenczy 2000, p
18-19; Szende
1994)}{Consonants , p 18-19(Siptár and Törkenczy 2000, p 18-19; Szende 1994)}}\label{con}}

\begin{longtabu} to \linewidth {>{\centering}X>{\centering}X>{\centering}X>{\centering}X>{\centering}X>{\centering}X>{\centering}X>{\centering}X>{\centering}X>{\centering}X>{\centering}X>{\centering}X>{\centering}X>{\centering}X}
\caption{\label{tab:hu_con}Consonants}\\
\toprule
  & Voicing & labial & O & alveolar & O & post-alveolar & O & palatal & O & velar & O & glottal & O\\
\midrule
nasal &  & m & m & n & n & ɲ & ny &  &  &  &  &  & \\
\cmidrule{1-14}
 & vs & p & p & t & t &  &  & c & ty & k & k &  & \\

\multirow{-2}{*}{\centering\arraybackslash stop} & vd & b & b & d & d &  &  & ɟ & gy & g & g &  & \\
\cmidrule{1-14}
 & vs & f & f & s & sz & ʃ & s &  &  &  &  & h & h\\

\multirow{-2}{*}{\centering\arraybackslash fricative} & vd & v & v & z & z & ʒ & zs &  &  &  &  &  & \\
\cmidrule{1-14}
 & vs &  &  & ts & c & tʃ & cs &  &  &  &  &  & \\

\multirow{-2}{*}{\centering\arraybackslash affricate} & vd &  &  & dz & dz & dʒ & dzs &  &  &  &  &  & \\
\cmidrule{1-14}
tap/trill &  &  &  & r & r &  &  &  &  &  &  &  & \\
\cmidrule{1-14}
approximant &  &  &  & l & l &  &  & j & j|ly &  &  &  & \\
\bottomrule
\multicolumn{14}{l}{\textsuperscript{a} placeless /N/ appears preconsonantally and receives place from the following consonant}\\
\multicolumn{14}{l}{\textsuperscript{b} The [O] columns stands for orthography.}\\
\multicolumn{14}{l}{\textsuperscript{c} [vs] and [vd] in the second column stands for voiceless and voiced}\\
\end{longtabu}

\hypertarget{v}{%
\subsection{Vowels (Szende 1994)}\label{v}}

\begin{longtable}[]{@{}lllll@{}}
\toprule
& IPA & Writing & Example & Gloss\tabularnewline
\midrule
\endhead
1 & \textipa{A} & a & agy & brain\tabularnewline
2 & a\textipa{:} & á & ágy & bed\tabularnewline
3 & \textipa{E} & e & egy & one\tabularnewline
4 & e\textipa{:} & é & ért & understand\tabularnewline
5 & i & i & irt & eradicate\tabularnewline
6 & i\textipa{:} & í & ír & write\tabularnewline
7 & o & o & orr & nose\tabularnewline
8 & o\textipa{:} & ó & ól & sty\tabularnewline
9 & \o & ö & öl & kill\tabularnewline
10 & \o\textipa{:} & ő & őr & guard\tabularnewline
11 & u & u & ujj & finger\tabularnewline
12 & u\textipa{:} & ú & úgy & like that\tabularnewline
13 & y & ü & ügy & affair\tabularnewline
14 & y\textipa{:} & ű & űrr & space\tabularnewline
Marginal\footnote{typically used for pronuncing letters in abbreviations
  and borrowed words (Siptár and Törkenczy 2000, p 280)} & a, e,
\textipa{O:}, \textipa{E:} & & &\tabularnewline
\bottomrule
\end{longtable}

\hypertarget{note-on-the-alphabet}{%
\subsection{Note on the Alphabet}\label{note-on-the-alphabet}}

Letters q, x, y, w as well as the combination `ch' are excluded in
hu.rules since they only occur in borrowed words like homoszexuális, and
traditional surnames e.g.~Andrássy.

\hypertarget{rules}{%
\section{Rules}\label{rules}}

see hu.rules written based on Section \ref{con} and \ref{v}

\hypertarget{controversies-and-choices}{%
\subsection{Controversies and choices}\label{controversies-and-choices}}

\begin{enumerate}
\def\labelenumi{\arabic{enumi}.}
\tightlist
\item
  Whether ``ty'', ``dy'' are affricates or stops as seen in \ref{con}
\end{enumerate}

\begin{itemize}
\tightlist
\item
  stops (see Siptár and Törkenczy 2000, p 82)
\item
  does not changed the result of translation
\end{itemize}

\begin{enumerate}
\def\labelenumi{\arabic{enumi}.}
\setcounter{enumi}{1}
\tightlist
\item
  Whether there are dipthongs
\end{enumerate}

\begin{itemize}
\tightlist
\item
  no (see Siptár and Törkenczy 2000, p 16-18)
\end{itemize}

\begin{enumerate}
\def\labelenumi{\arabic{enumi}.}
\setcounter{enumi}{2}
\tightlist
\item
  Whether {[}dz{]} is a phoneme or just a surface sound
\end{enumerate}

\begin{itemize}
\tightlist
\item
  surface (see Siptár and Törkenczy 2000, p 87-89)
\item
  underlying (see Szende 1994)
\end{itemize}

\begin{enumerate}
\def\labelenumi{\arabic{enumi}.}
\setcounter{enumi}{3}
\tightlist
\item
  The placeless {[}N{]} as the underlying default nasal preconsonantally
  that undergoes progressive nasal assimilation vs.~{[}m{]}, {[}n{]} as
  the same with surface and underlying forms
\end{enumerate}

\begin{itemize}
\tightlist
\item
  (see Siptár and Törkenczy 2000, p 207-212)
\item
  decide on keeping {[}m{]} and {[}n{]} as underlying forms due to words
  such as háromdimenziós which is három+dimenziós. m keeps its place in
  this example.
\end{itemize}

\hypertarget{lenition-processes}{%
\section{Lenition Processes}\label{lenition-processes}}

\begin{itemize}
\tightlist
\item
  Degemination (Siptár and Törkenczy 2000, p 83): Across word
  boundaries, affricates remain unmerged in careful speech. In
  colloquial speech, the first affricate may lenite into a fricative due
  to OCP
\end{itemize}

\begin{longtable}[]{@{}llll@{}}
\toprule
& Writing & careful speech & colloquial speech\tabularnewline
\midrule
\endhead
& rác cég & {[}raːts-tseːg{]} & {[}raːs-tseːg{]}\tabularnewline
& bölcs csere & {[}bøltʃ-tʃɛrɛ{]} & {[}bølʃ-tʃɛrɛ{]}\tabularnewline
\bottomrule
\end{longtable}

\hypertarget{references}{%
\section*{References}\label{references}}
\addcontentsline{toc}{section}{References}

\hypertarget{refs}{}
\leavevmode\hypertarget{ref-ST}{}%
Siptár, Péter, and Miklós Törkenczy. 2000. \emph{The Phonology of
Hungarian}. Oxford University Press.

\leavevmode\hypertarget{ref-IPA}{}%
Szende, Tamás. 1994. ``Hungarian.'' \emph{Journal of the International
Phonetic Association} 24 (2): 91--94.


\end{document}
